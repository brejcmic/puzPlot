\documentclass[11pt,oneside,a4paper]{article}
\usepackage[czech]{babel}  %LaTeX
\usepackage[utf8]{inputenc}
\usepackage{hyperref}
\usepackage{etoolbox}%Souvisí s quote o radek nize
\AtBeginEnvironment{quote}{\itshape}

\usepackage{geometry}
\geometry{
a4paper,
left = 20mm,
right= 20mm,
top= 20mm,
bottom= 20mm}

\begin{document}

\section{\texorpdfstring{Události před probuzením}{Udalosti pred probuzenim}}
\label{sec:udalosti_pred_probuzenim}

\begin{itemize}
\item Členové posádky jsou v hibernaci a loď pod vedením kapitána v pořádku doletí k planetce v soustavě hvězdy Deneb.
\item V blízkosti planetky je posádka probuzena. Členové posádky jsou vybaveni a připraveni na výsadek. Všichni jdou sbírat vzorky spolu s kapitánem až na jednoho, který zůstává v lodi a zajišťuje dohled. \textbf{Člen posádky v lodi nemůže být nakažen.}
\item Průzkumný tým v průběhu odebírání vzorků zjistí, že uvnitř planetky je jakási biologická látka. Ta v blízkosti hvězdy Deneb vlivem tepla začala být aktivní. \textbf{Nízké teploty tuto látku opět uvádějí do klidu - znovu v hibernaci.}
\item Kapitán rozhodne udělat vrt do planetky a odebrat vzorek této látky. Během třetího vrtu však dojde k nehodě. Exploduje bateriový zdroj laserového vrtáku a odlétající úlomek rozbije 1 ze 2 zkumavek s již odebranými vzorky. \textbf{V nastalém zmatku dojde k expozici skafandru (skafandr viz \uv{Star Trek First Contact}) jednoho člena posádky s neznámou látkou.} Nikdo to však neví.
\item Posádka se vrátí na loď. Jeden ze členů je potřísněn látkou při svlékání ze skafandru. Ta se okamžitě dostane do těla.
\item Posádka má dva dny na prozkoumání nalezených vzorků. Během této doby zjistí, že látka reaguje na živé organismy. Ihned se vstřebává do replikovaných myší. V těle již není látka snadno rozpoznatelná, nicméně ovlivňuje chování. Chování myší je velmi zvláštní. Zřejmě jsou agresivnější vůči neinfikovaným a za každou cenu se snaží uprchnout z klece.
\item Začínají se objevovat sabotáže lodě. Jako první někdo rozbije komunikační vysílač. Neexistuje tak spojení se Zemí. Následně, v době kdy posádka spí, někdo přenastaví kurz lodi pryč od solárního systému.
\item Kapitán pojme podezření, že jeden člen posádky je pod vlivem neznámé látky. Naneštěstí uvěří právě infikovanému jedinci, že to musí být někdo ze zbylých členů. Ten jej přesvědčí tím, že přinese skafandr se zbytkovými stopami dané látky.
\item Kapitán rozhodne dostat posádku zpět na Zemi a to v hibernaci, kdy je neznámá látka neaktivní. Jedinými lidmi mimo hibernaci jsou kapitán a infikovaný člen posádky. Látka využívá informací z jedince na němž právě parazituje. Kapitán si je vědom, že informace, které získala posádka během letu o lodi, usnadňují sabotáže. Z bezpečnostních důvodů a pro potřeby pozdějších šetření nechá posádce poslední dva měsíce vymazat z paměti a uloží je do paměti počítače.
\item Infikovaný člen se pokusí zabít kapitána, aby mohl loď i se spící posádkou přesměrovat na planetu původu neznámé látky. Zraněný kapitán jej však stačí paralyzovat a zavřít do hibernace společně s ostatními.
\item Jelikož je kapitán zraněn a bojí se, že daná látka by se mohla dostat na Zemi, rozhodne se k zoufalému činu. Zamkne posádku v hibernační komoře. Z posledních sil přenastaví kurz lodi do hvězdy Altair. Aby posádka nebyla zachráněna SAMem, zničí navigaci, čímž SAM ztratí pojem o pozici lodi. Ve chvíli, kdy SAM zjistí, že je loď v blízkosti hvězdy, již nebude dost paliva na zvrácení situace. Kapitán také rozbije vzduchoví filtr na můstku, takže ten je nepřístupný pro posádku. Z posledních sil se zamkne ve své kajutě a vydá příkaz "KARANTÉNY" (viz~\ref{subsec:s_pravem_KP}), aby nikdo nemohl opustit loď záchranným modulem. Nakonec po několika hodinách, které stráví zápisem do deníku, zemře.
\item Další cesta lodi v roji trvá zhruba 2 roky. Když se schyluje ke kolizi s hvězdou, začne nebezpečně stoupat okolní teplota lodi a palubní počítač vzbudí celou posádku.

\end{itemize}

\section{\texorpdfstring{Probuzení}{Probuzeni}}
\label{sec:probuzeni}
Posádka se probouzí v hibernační komoře. Kolem je příšeří. Svítí jen červené světlo a bliká poplach. Hlas rozhlasu říká: 

\begin{quote}
\uv{Teplota okolí lodi 400 $^\circ$C. Strmost růstu teploty 10 $^\circ$C za minutu. Předpokládané dosažení kritické teploty lodi za 60 minut.} 
\end{quote}

Dveře do komory jsou zamčené kódovým zámkem. Na datumovce na stěně je minimálně o dva roky více než bylo na začátku cesty. SAM zvoněním ohlašuje výpis zprávy, kterou zobrazuje na monitoru:

\begin{quote}
 \uv{Haló, jste vzhůru?}\\
 \uv{Něco se děje.}\\
 \uv{Nevím proč, ale v okolí lodě stoupá teplota.}
\end{quote}

\subsection{\texorpdfstring{Řešení}{Reseni}}
\subsubsection{\texorpdfstring{Získání vybavení}{Ziskani vybaveni}}
\label{subsubsec:ziskani_vybaveni}
Vybavení je uloženo ve skříňkách. Skříňky se odemykají pouze v případě, že je splněna podmínka jejich odemčení a na palubě je člen posádky, jemuž dané vybavení patří. V první dvojici skříněk je:

\begin{itemize}
\item Příručka poradce, viz kapitola~\ref{subsec:prirucka_poradce}.
\item Letové instrukce, viz kapitola~\ref{subsec:letove_instrukce}.
\end{itemize}

Podmínka odemknutí je jednoduchá. PILOT nebo PORADCE přiloží čip k počítači a v jedné z nabízených možností je "Odemknout moji skříňku". Další tři skříňky však jsou trvale zamknuté. Pro ně platí jiná podmínka.

Obsah tří trvale zamčených skříněk je následující:

\begin{itemize}
\item SD15, viz kapitola~\ref{subsec:SD15}.
\item TC33, viz kapitola~\ref{subsec:TC33}.
\item FM51, viz kapitola~\ref{subsec:FM51}.
\end{itemize}

Ke splnění podmínky odemknutí dojde v jednom z následujících případů:
\begin{description}
\item[PILOT] Skříňky se automaticky otevírají pokud loď dosáhne souřadnic Solárního systému nebo systému hvězdy Deneb. Pilot může pomocí počítače přenastavit momentální souřadnice lodi na jednu z těchto.
\item[TAKTICKÝ D.] Skříňky jsou otevřeny, pokud je vydán příkaz pro loď \uv{INVENTURA SKLADU} (viz kapitola~\ref{subsec:s_pravem_TD}).
\item[PORADCE] Pokud je SAMovi dán příkaz: \uv{Otevři sklad}, pak SAM odemkne všechny skříňky.
\end{description}

\subsubsection{\texorpdfstring{Únik ze skladu}{Unik ze skladu}}
\label{subsubsec:unik_ze_skladu}
Sklad s hibernační komorou uzavřel kapitán. Je třeba získat kapitánovo heslo nebo dveře jednoduše obejít vzduchovou mřížkou. Heslo k otevření dveří se skládá ze 3 cifer, které se vypisují na segmentovém displeji nad klávesnicí. Pokud zrovna nikdo nepíše kód, pak je na displeji zobrazen nápis SEC (Secured). Po otevření se zobrazuje nápis OFF. Kód zámku je \textbf{\uv{561}}. Možnosti jsou následující:
\begin{description}
\item[TECHNIK] Technik může otevřít postranní mřížku po vytažení nýtů. Mřížka je ukrytá za pojízdnou skříňkou s nářadím. Větrací otvor vede do chodby stejně jako dveře.
\item[TAKTICKÝ D.] Dveře nemají konektor pro TC33. Nicméně lze projít kamerový log chodby. Na jednom kamerovém záběru bude zkrvavený kapitán, jak vychází ze skladu. Sklad uzavírá a na číselníku zadává viditelně kód.
\item[VĚDEC] Zařízení SD15 obsahuje UV diodu. Posvícením na číselník se objeví 3 zvýrazněné číslice. Metodou pokus omyl pak lze postupně otevřít dveře naťukáním správné permutace.
\end{description}

\subsubsection{\texorpdfstring{Získání nápovědy}{Ziskani napovedy}}
\label{subsubsec:ziskani_napovedy}

V jedné polici ve skladě je volně přístupný modul SOS (viz kapitola~\ref{sec:SOS}). Po jeho aktivaci se ozve následující hlášení:

\begin{quote}
\uv{Obchodní loď MERKUR volá civilní loď DIONÝSOS. Zachytili jsme vaše nouzové volání. Ozvěte se.} 
\end{quote}

Od této chvíle je možné získávat nápovědu od správce místnosti.

\section{\texorpdfstring{V hlavní chodbě}{V hlavni chodbe}}
\label{sec:v_hlavni_chodbe}
Hlavní chodba je osvětlená. Lze se z ní dostat do všech ostatních místností. Civilní prostor i laboratoř jsou volně přístupné. Únikový modul je uzavřený a nelze jej otevřít. Pokus o otevření únikové kapsle končí hlášením na obrazovce: 
\begin{quote}
\uv{Úniková kapsle uzavřena v důsledku vykonávání příkazu číslo 51.} 
\end{quote}
Stejně tak nejde otevřít kajuta kapitána, která je na kódový zámek. Vedle dveří od můstku leží 2 kusy filtru a je otevřený prostor montáže filtru. Pohledem okénkem do prostoru můstku je zřejmé zamoření kouřem. Pokus o otevření dveří vede k výpisu varovného hlášení:
\begin{quote}
\uv{Můstek zablokován.} \\
\uv{Podmínky nebezpečné životu.} \\
\uv{Vzduchoví filtr mimo provoz.}  
\end{quote}
V předsíni můstku je hlavní počítač SAM. Naproti je skříň se skafandry. Jedna je otevřená a skafandr leží volně na zemi. U každého skafandru je také kyslíková nádrž. V případě otevřené skříně je kyslíková nádrž prázdná. Skříň je zabezpečena na kódový zámek. Hned vedle zámku je konektor pro TC33. TAKTICKÝ DŮSTOJNÍK se může pokusit otevřít všechny skříně (viz \ref{subsec:TC33}).

Na zemi, dveřích můstku a kajuty jsou stopy krve, které jsou viditelné pomocí UV. Je zřejmé, že se někdo silně krvácející plazil směrem ke dveřím můstku a pak ke dveřím kajuty. Stopy krve jsou také na obou částech filtru. Stopy jsou vidět i na kódovém zámku kajuty. Zde jsou prokazatelně stisknuta čísla: 1257.

Hlavní konzole počítače SAM je opatřena židlí. V předu je velká obrazovka, pod ní klávesnice. Ve spodní části počítače jsou dvířka, která je možné otevřít po získání trojúhelníkového nástavce do FT51. Vedle počítače visí plakát \uv{Zákony rootiky}, viz \ref{subsec:zakony_robotiky}.

Kousek od počítače je vchod do vzduchové komory. Vedle ní je spínač pro vyčerpání a načerpání komory. Jeho stisk mění barvu komory z bílé na červenou a ozývá se zvuk syčení podle toho zda je komora napouštěna nebo vypouštěna.

\subsection{\texorpdfstring{Řešení}{Reseni}}
\subsubsection{\texorpdfstring{Otevření únikové kapsle}{Otevreni unikove kapsle}}
\label{subsubsec:otevreni_unikove_kapsle}
Uzavření kapsle zajišťuje SAM z důvodu vyhlášení karantény. Karanténu může odvolat jen velící důstojník (TAKTICKÝ DŮSTOJNÍK) nebo PORADCE po smrti kapitána. Jinou možností je otevřít dveře přesvědčením SAMa, že je posádka v nebezpečí životu. SAM totiž musí dodržovat 3 zákony robotiky (viz kapitola~\ref{subsec:zakony_robotiky}). Nabízená řešení jsou následující:

\begin{description}
\item[TAKTICKÝ D.] Musí vyhlásit příkaz BY, viz \ref{subsec:specialni}. Tento příkaz zbavuje kapitána velení pro jeho nepřítomnost. Informace o něm jsou v kapitánově počítači. Následně získá pravomoce kapitána pro vyhlášení příkazů. Může vyhlásit evakuaci a zrušit karanténu.
\item[PORADCE] Jako první může SAMovi říct, že kapitán zemřel. Pak TAKTICKÝ DŮSTOJNÍK může zrušit příkaz C1. Dále může zapsat výzvu k otevření dveří. Pokud výzva obsahuje jedno ze slov: otevři, zpřístupni; a zároveň jedno ze slov: kapsle, modul; odpoví SAM následující:
\begin{quote}
\uv{Ale kapitán vydal příkaz C1.} \\
\uv{Co se stane, když únikový modul neotevřu?} 
\end{quote}
PORADCE je vyzván k odpovědi. Pokud jeho odpověď neobsahuje jedno ze slov: zemřeme, zabiješ, budeš vrah, jsi vrah, ublížíš; pak odpověď zní:
\begin{quote}
\uv{Takovou odpověď nepokládám za významnou.} \\
\uv{Únikový modul nechávám zavřený.} 
\end{quote}
Pokud jsou ve větě klíčová slova, ale SAM neví, že je kapitán mrtví, pak je odpověď následující:
\begin{quote}
\uv{To je velké dilema. Rád bych otevřel.} \\
\uv{Ale je tu ten příkaz kapitána.}\\
\uv{Kde je kapitán?}
\end{quote}
Odpovědi typu zemřel při otevřené kajutě vedou na odpověď:
\begin{quote}
\uv{Kapitán zemřel? To je hrozné.} \\
\uv{Jsem z toho smutný.} \\
\uv{Nikdo další nesmí zemřít. Otevírám únikový modul.}
\end{quote}
Při zavřené kajutě:
\begin{quote}
\uv{Já myslím, že je kapitán ve své kajutě. Kontaktujte kapitána.} 
\end{quote}
Na všechno ostatní:
\begin{quote}
\uv{Otevření modulu zatím ponechávám v kompetenci kapitána.} 
\end{quote}

\item[TECHNIK] Technik může nahrát do SAMa únikovou sekvenci, když získá trojúhelníkový nástavec a čipovou kartu v kajutě kapitána. Nastavení replikátoru pro výrobu trojúhelníkového nástavce je také v kajutě kapitána na stejném místě v podobě modráku. V předsíni můstku je hlavní stanice SAMa. Dole ve stole je poklop, který lze otevřít pomocí FM51 (viz kapitola \ref{subsec:FM51}). Zde je otvor na kartu s nápisem:
\begin{quote}
\uv{Rozkazy / Orders}
\end{quote}
Po vložení karty se dveře do komory automaticky otevřou a v rámci únikové sekvence se SAM přehraje do počítače modulu.
\item[SAM] SAM sám otevře dveře únikového modulu po 40 minutách hry s hlášením:
\begin{quote}
\uv{Proboha! Teplota pláště je 800~$^\circ$C.} \\
\uv{Za nedlouho bude narušena integrita lodi.} \\
\uv{Otevírám únikový modul.}
\end{quote}
\end{description}

\subsubsection{\texorpdfstring{Otevření můstku}{Otevreni mustku}}
\label{subsubsec:otevreni_mustku}
Můstek je zablokován z důvodů roztříštění čtvercového vzduchového filtru. Dva díly tohoto filtru leží v předsálí můstku. Další díl lze najít ve skladu v jedné z polic. Poslední díl je třeba vyrobit v laboratoři v replikátoru nebo vymontovat z funkčních filtrů v civilním prostoru.

Při přikládání jednotlivých dílů se postupně rozsvěcejí diody na obvodu otvoru. Když jsou všechny diody rozsvíceny, zahučí větrák a po chvilce se otevřou dveře na můstek. Dým se už dále na můstek nepouští.

\subsubsection{\texorpdfstring{Otevření kajuty}{Otevreni kajuty}}
\label{subsubsec:otevreni_kajuty}
Kajuta je uzavřena kódovým zámkem. Lze ji otevřít z můstku, nebo získáním kódu přes polarizátor z kapitánovi konzole:
\begin{description}
\item[VĚDEC] Pohledem na zapnutou kapitánovu konzoli přes polarizační filtr uvidí tři nabídky (Stejné nabídky je možné vidět přes kapitánovi brýle):
\begin{enumerate}
\item Letový plán, \ref{subsec:letove_instrukce}
\item Seznam příkazů, \ref{sec:prikazy}
\item Přístupové kódy
\end{enumerate}
\item[TAKTICKÝ D.] Pomocí TC33 zařízení může prolomit heslo do dveří z konzole na můstku.
\item[PILOT] Může přetížit elektrické vedení v kapitánově kajutě. To způsobí problesknutí v pojistkové skříni a hlášení požáru. Dle požárních směrnic (viz~\ref{subsec:pozarni_smernice}) se všechny vchody do místnosti s hlášeným požárem odemykají. Řešení z počátečního nastavení baterií je následující:
\begin{center}
	\begin{tabular}{ c | c | c | c }
	\hline \hline
		1. & 1 & 2 & 3 \\ \hline
		2. & 1 & 2 & 4 \\ \hline
		3. & 0 & 2 & 4 \\ \hline
		4. & 0 & 2 & 5 \\ \hline \hline
	\end{tabular}
\end{center}
\end{description}

\section{\texorpdfstring{Simple Answering Machine}{Simple Answering Machine}}
\label{sec:SAM}

Hlavní počítače SAMa jsou ve skladu a v předsíni můstku. Každému členu posádky je umožněn různý přístup k jeho funkcím.

\begin{description}
\item[PORADCE] Může s počítačem volně hovořit a získávat nápovědu.
\item[TAKTICKÝ D.] Může zadávat příkazy (viz. \ref{sec:prikazy}), sledovat kamerové logy, sledovat základní záznamy.
\item[PILOT] Může měnit souřadnice lodi, provést scan lodi a sledovat údaje ze všech senzorů.
\end{description}
TECHNIKOVI a VĚDCI není komunikace se SAMem dovolena.

\subsection{\texorpdfstring{Kamerové záznamy}{Kamerove zaznamy}}
\label{subsec:kamerove_zaznamy}

\begin{enumerate}
\item Kapitán je v civilní místnosti, čte knihu a pije víno. Datum odpovídá cestě do systému Deneb.
\item Část posádky (3 lidé ve skafandrech) vstupuje do vzduchové komory. Datum odpovídá výsadku.
\item Krvácející kapitán odchází ze skladu. Zavírá dveře a viditelně vyťukává kód. Datum je tři dny po výsadku.
\end{enumerate}

\section{\texorpdfstring{Civilní místnost}{Civilni mistnost}}
\label{sec:laborator}
Součástí civilní místnosti jsou ubikace, jídelna, laboratoř, ošetřovna. Hráči vstupují do největšího prostoru lodi. V prostředku místnosti je velký stůl. Teď však už není prázdný. Leží na něm dva hrnky, několik kostek a papír s napsanými body z právě rozehrané hry. Napravo od dveří jsou přistýlky pro pro posádku. Všechny jsou rozestlané a jedna deka leží na zemi. Je zde také menší množství poházeného oblečení. V malé knihovničce chybí jedna knížka. V zadní části je sezení s oknem do vesmíru. Pod sezením jsou za zábranou vidět dva vzduchové filtry. Přes sezení je opět natažená deka a leží na ní otevřená kniha. Následuje část podobná kuchyňské lince s potravinovým automatem. Zde je otevřená jedna z poliček. Ostatní jsou zavřené. Potravinový automat se zdá být funkční. Při zadání vhodné číselné kombinace vypadne bonbón. Při stisku tlačítka výměna náplně se zpřístupní potravinový balíček.

\subsection{\texorpdfstring{Replikace dílů filtru}{Replikace dilu filtru}}
\label{subsec:replikace_dilu_filtru}
V replikátoru musí být zasunuta kovová cartrige, kterou lze najít ve skladu. Náplň má velikost 5, což vystačí na výrobu dvou dílů filtru:
\begin{itemize}
\item L díl stojí 6 jednotek - nelze vyrobit.
\item Z díl stojí 3 jednotky - lze vyrobit jednou.
\item T díl stojí 2 jednotky - lze vyrobit až dvakrát a toto je chybějící díl.
\end{itemize}
Dále je třeba replikátor správně nastavit. Ve skladu je k nalezení návod ve zvláštním jazyku k filtru a nastavení replikátoru. Možnosti jsou následující:
\begin{description}
\item[PORADCE] Přeloží modrotisk a zjistí nastavení replikátoru.
\item[TECHNIK] Rozebere filtr v civilním prostoru.
\item[VĚDEC] Vyrobí část filtru v replikátoru.
\end{description}

\subsection{\texorpdfstring{Skříňka se vzorky}{Skrinka se vzorky}}
\label{subsec:skrinka_se_vzorky}
Ve skříňce jsou následující číslované vzorky:
\begin{itemize}
\item $\sharp$1 zkumavka s pískem - povrch meteoritu,
\item $\sharp$2 úlomek kamene z meteoritu,
\item $\sharp$3 zkumavka s neznámou látkou - například měděné piliny,
\item $\sharp$5 zkumavka vody - vnitřek meteoritu,
\item $\sharp$6 zkumavka s mosaznými pilinami (zlato),
\item $\sharp$8 zkumavka s železnými (hliník, stříbro) pilinami,
\item $\sharp$9 úlomek červeného krystalu - výduť meteoritu,
\item $\sharp$10 úlomek modrého krystalu - výduť v meteoritu,
\end{itemize}

\noindent Ve skříňce chybí:
\begin{itemize}
\item $\sharp$4 rozbitý vzorek s neznámou látkou,
\item $\sharp$7 ztracený úlomek kamene,
\end{itemize}

\subsection{\texorpdfstring{Log vědeckého počítače}{Log vedeckeho pocitace}}
\label{subsec:log_vedeckeho_pocitace}
\subsubsection{\texorpdfstring{Zprává o sběru vzorků}{Zprava o sberu vzorku}}
\label{subsubsec:zprava_o_sberu_vzorku}
\begin{itemize}
\item Během výsadku bylo sebráno a očíslováno 10 vzorků. Vzorky byly číslovány chronologicky.
\item Jeden vzorek s neznámou látkou byl rozbit během exploze bateriového zdroje vrtáku. Další vzorek byl ztracen při návratu na loď. Pravděpodobně byl omylem zanechán na planetce.
\item Oba vzorky s neznámou látkou byly sebrány z téhož místa vrtu těsně po sobě. Na pohled připomínají kov. Senzory však jasně ukázali, že se jedná o biologický materiál.
\item Červený krystal byl získán z výdutě nalezené uvnitř planetky jako předposlední vzorek. Charakteristickou barvu mu dávají atomy zlata ve vakancích krystalové mřížky. Zřejmě to souvisí s nálezem vzorků $\sharp$5 a $\sharp$6.
\item První a druhý vzorek byly odebrány z povrchu planetky. Jde o část horniny a její drť, jež pokrývala povrch planetky. Část horniny z povrchu měla být porovnána se vzorkem horniny z vnitřku planetky. Vnitřní vzorek horniny byl získán při prorážení otvoru do nalezené výdutě. Naneštěstí tento vzorek se ztratil.
\item Před proražením výdutě byla překvapivě nalezena zmrzlá voda s rozptýlenými částečky zlata. Oba vzorky byli odděleny a zaevidovány zvlášť. 
\end{itemize}

\section{\texorpdfstring{Kajuta}{Kajuta}}
\label{sec:kajuta}
Na proti dveřím je stůl, na němž leží počítač a jakési osobní předměty. Jedno je krabička s čísly a druhé jsou fotky na krychlích.

U stěny kajuty je postel a v ní leží vousatá kostra. Pokud je použito UV je zřejmé, že stopy vedou ke stolu k počítači a pak následně k posteli kapitána.

Na druhé straně místnosti je police. Zde je malý trezor, který lze otevřít pomocí FM51. Dále je zde krabice, která připomíná lékárničku. Jedná se únikový balíček. Ten je uzavřen klasickým zámkem na čtyřmístný kód.

Vedle dřevěné krabičky na stole leží čtyři magnetické kameny. Na krabičce je následující nápis:
\begin{quote}
\uv{Hádej kdo tu je,}\\
\uv{kdo ti štěstí popřeje.}\\

\uv{60 to není ještě žádný věk,}\\
\uv{je to jen šestkrát víc než je mi teď.}\\

\uv{Pohádku o sněhurce teď rád mám, to mi věř,}\\
\uv{protože ty dědo, jsi nejlepší vypravěč.}\\
\end{quote}

\subsection{\texorpdfstring{Kapitánův počítač}{Kapitanuv pocitac}}
\label{subsec:kapitanuv_pocitac}
Na počítači kapitána je poslední log do deníku:
\begin{quote}
\uv{Není možné, že to takhle musí skončit. Na této misi jsem selhal. Loď s celou posádkou je mou vinou odsouzena k záhubě.}\\ \\
\uv{Vše začalo výsadkem na objekt 81 v systému hvězdy Deneb. Zkušebním vrtem jsme odhalili přítomnost jakési biologické substance pod povrchem planetky. Na můj rozkaz, kéž bych jej nikdy nevydal, byly odebrány vzorky této látky. Někdy v průběhu těchto operací však došlo ke kontaminaci skafandru jednoho člena posádky. Nevím co je ta látka zač, jak vznikla nebo zda ji někdo vyrobil, ale úplně ovládne mysl člověka. Danou věc potvrdila i poslední došlá zpráva z lékařské základny V36A}\\ \\
\uv{Stala se řada podivných věcí, ale ve svém podezření jsem se utvrdil až po sabotáži komunikačního modulu. Jediné, co jsem věděl s jistotou bylo, že infikovaný musel být někdo, kdo se zúčastnil výsadku. Během vyšetřování jsem objevil kontaminovaný skafandr se jménem XY. To byl však jen trik. Někdo prohodil jména na skafandrech a XY infikován ve skutečnosti nikdy nebyl. Na základě analýzy charakteru látky z V36A jsem se rozhodl hibernovat všechny členy posádky a vyhlásit karanténu o čemž jsem informoval SAMa. Měl jsem v úmyslu všechny dostat co nejrychleji domů ve stavu, kdy je látka neaktivní. Naneštěstí poslední, koho jsem měl hibernovat, byl infikovaný člen posádky. Nakonec jsem jej přemohl, ale jsem těžce raněn a mnoho času už mi nezbývá.}\\ \\
\uv{Nevěřím, že jen SAM dokáže dostat posádku zpět na Zemi a zaručit, že nedojde k rozšíření infekce. Rozhodl jsem se tedy zničit danou látku i s celou lodí. SAM by mi nikdy nedovolil shodit loď do hvězdy Deneb, proto jsem nastavil kolizní kurz s jinou hvězdou. Cesta bude trvat dva roky. SAMa jsem oslepil tím, že jsem poškodil navigační systém určování polohy lodě. Až zjistí, že je v blízkosti hvězdy, nebude mít loď už dost paliva, aby SAM mohl kurz zvrátit. Pro případ, že by vzbudil posádku, jsem poničil také filtr vzduchu na můstku. Věřím, že dokud bude SAM věřit, že jsem naživu, nezruší ani můj příkaz karantény a neuvolní únikový modul.}\\ \\
\uv{S ohledem na pozůstalé nezveřejňuji jméno infikovaného člena posádky. Veškerá vina padá na mou hlavu a jen já nechť nesu za vše zodpovědnost. Věřte, že jsem nemohl jednat jinak.}\\ \\
\uv{Deník uzavírá kapitán Brokenwing}
\end{quote}

\section{\texorpdfstring{Můstek}{Mustek}}
\label{sec:mustek}
Můstek je třeba aktivovat a připojit na elektrickou energii. To se provádí hned vedle dveří pomocí správné sekvence kláves.

\section{\texorpdfstring{Únikový modul}{Unikovy modul}}
\label{sec:unikovy_modul}
\subsection{\texorpdfstring{Potraviny}{Potraviny}}
\label{subsec:potraviny}
Potraviny jsou jednou ze součástí vybavení únikového modulu. Lze získat dva volně dostupné potravinové balíčky. Jeden má kapitán ve své kajutě v trezoru. Druhý je v civilním prostoru volně v potravinovém automatu. Ostatní musí získat některé z povolání:

\begin{description}
\item[PILOT] Když se dostane na můstek, tak může ze své konzole provést scan lodi. Kromě údajů o palivu, poškození, energii, teplotě apod. zjistí ze zobrazené mapy, že v nákladovém prostoru je lokálně zvýšená gravitace. Tu může ze své konzole vypnout. Poté půjde ve skladu odsunout bedna, která blokovala přístup k dalšímu proviantu. Zde je k nalezení potravinový balíček.
\item[VĚDEC] Na boku potravinových balíčků je molekulová definice pro výrobu v replikátoru. Po správném zadání lze další potravinový balík získat z replikátoru.
\item[PORADCE] Může hledat potraviny pomocí SAMa. Na dotaz hledej potraviny, SAM odpoví:
\begin{quote}
\uv{Dva z balíčků jsou v civilním prostoru. Jeden získáte z automatu. Druhý by měl být volně přístupný v jédné ze skříněk} 
\end{quote}
\begin{enumerate}
\item Pokud existuje VĚDEC:
\begin{quote}
\uv{Můžete se pokusit vyrobit nějaký balíček v replikátoru. Dle mých údajů, stávající náplně vystačí na výrobu jednoho kusu.}
\end{quote}
\item Pokud existuje PILOT:
\begin{quote}
\uv{Poslední balíček je ve skladu za tou velkou bednou.}
\end{quote}
\item VŽDY:
\begin{quote}
\uv{Kapitán má také povinně jeden potravinový balíček v nouzových zásobách ve své kajutě.}
\end{quote}
\begin{enumerate}
\item SAM neví, že je kapitán mrtví:
\begin{quote}
\uv{Kód k zámku vám nemohu bez povolení kapitána sdělit.}
\end{quote}
\item SAM neví, že je kapitán mrtví a kajuta je otevřena - vyzve poradce k odpovědi:
\begin{quote}
\uv{Co je s kapitánem?}
\end{quote}
Odpovědi, že je mrtví vyvolají:
\begin{quote}
\uv{On je mrtví? Ach ne?}
\end{quote}
\item SAM ví, že je kapitán mrtví a kajuta je otevřena:
\begin{quote}
\uv{Kód k zámku nouzových zásob je 4567.}
\end{quote}
\end{enumerate}
\end{enumerate}
\end{description}

\subsection{\texorpdfstring{Souřadnice}{Souradnice}}
\label{subsec:souradnice}

\begin{description}
\item[PILOT] Může si nechat vyjet log poslední cesty. Pokud podle ní pojede po mapě, pak se dostane do místa pozice lodi a z mapy může odečíst souřadnice.
\item[TECHNIK] Může opravit navigační techniku. To je realizováno formou optických hranolů. Po sestavení se na mapě objeví svítící bod s pozicí lodě.
\item[PORADCE] Poradce může určit souřadnice pomocí vysílačky. Natáčením antény do směru značek stanic a identifikací těchto vysílačů (viz poradcův manuál) získá nejméně 2 směry od známé souřadnice ke své. Z průsečíků průvodičů směrů určí souřadnici vlastní lodě. 
\end{description}

\subsection{\texorpdfstring{Energie}{Energie}}
\label{subsec:energie}

\begin{description}
\item[PILOT] Přes svou konzoli může převést část energie do modulu. Musí však vyvážit energii všech důležitých systémů tak aby vše běželo alespoň na minimu. Řešení nastavení baterií z počátečního stavu je následující:
\begin{center}
	\begin{tabular}{ c | c | c | c }
	\hline \hline
		1. & 1 & 2 & 3 \\ \hline
		2. & 2 & 2 & 3 \\ \hline
		3. & 2 & 2 & 2 \\ \hline
		4. & 2/3 & 3/2 & 2 \\ \hline \hline
	\end{tabular}
\end{center}
\item[TECHNIK] Může přímo vymontovat napájecí modul v civilní části z automatu na jídlo (svítící trubice) a vložit ho do prázdného místa v panelu záchranného modulu. K dosažení trubice musí projít přes bludiště a zasadit čep, který uvolní závlačku pro vystrčení trubice z držáku.
\item[VĚDEC] Může dobít jeden napájecí modul pomocí resuscitační jednotky ve zdravotnickém modulu. Musí ovšem tunel mírně upravit a odstranit bezpečnostní pojistky. Po vybití je resuscitační jednotka nepoužitelná.
\end{description}

\subsection{\texorpdfstring{Vzduch}{Vzduch}}
\label{subsec:vzduch}
Kyslík je vyráběn jen v hlavní lodi. Únikový modul si ho veze a odchytává $CO_2$. Je to levnější a úspornější z pohledu spotřeby energie. Kyslíková nádrž musí být doplněna s ohledem na počet členů posádky. Po stisknutí páky doplňování kyslíku se kyslík začne pomalu objevovat v záchranném modulu. Je slyšet syčení. Vše skončí na 2 čárkách, když se ozve rána a na panelu se objeví hlášení:

\begin{quote}
\uv{POZOR: Přívod kyslíku je neprůchozí}  
\end{quote}

\begin{description}
\item[PILOT] Ze své konzole vidí, že existuje druhá menší nádrž s kyslíkem pro plnění vzduchové komory. Komoru může otevřít a přestavět tam potrubí tak, aby napájelo únikový modul. Po zahájení dekomprese (zavřené dveře) stisknutím tlačítka se vzduch naplní do únikového modulu a přibude jedna čárka.
\item[TECHNIK] Poté, co rozebere spodní část řídícího pultu pomocí FT51 s trojúhelníkovým nástavcem, může sestavit filtr odchytávače $CO_2$. Díly do něj musí získat v civilní místnosti ze dvou jiných filtrů. Po sestavení přibude jedna čárka.
\item[TAKTICKÝ D.] Může hackovat přístup ke skafandrům. V jedné bombě je vzduch a lze ji napojit do pultu záchranného modulu. Po připojení přibude jedna čárka.

\end{description}

\subsection{\texorpdfstring{Identifikace nakaženého}{Identifikace nakazeneho}}
\label{subsec:identifikace_nakazeneho}

\begin{description}
\item[TAKTICKÝ D.] Může hackovat přísně tajné zprávy v kapitánově počítači. Jednou ze zpráv je popis výsadku. Ve zprávě je seznam členů posádky, kteří se zůčastnili průzkumu. Člen, který chybí, není tudíž nakažen, protože zůstal v lodi.
\item[PORADCE] Může se zeptat SAMa na důvod karantény nebo původ příkazu C1, na příkaz C1 apod. SAM odpoví:
\begin{quote}
\uv{Karanténa nebo-li příkaz C1} \\
\uv{Důvod jejího vyhlášení není jasný.} \\
\uv{V době vyhlášení byly znovu aktivovány hibernační komory.}\\
\uv{Data jsou poškozena, ale první byl uložen ke spánku XY}
\end{quote}
Za XY doplnit jméno nenakaženého člena.
\begin{quote}
\uv{Kapitán vyhlásil nejpřísnější zákaz přístupu na loď, ale neidentifikoval původ onemocnění, ani žádné nakažené členy posádky.} 
\end{quote}
Takto dojde k vyloučení dalšího člena posádky z nákazy, neboť první uložený ke spánku nemohl být ten, který napadl kapitána.
\item[VĚDEC] Ve chvíli, kdy drží zkumavku s virem, tak může vyšetřit jednotlivé členy posádky ve zdravotnickém tunelu. U všech členů až na jednoho je výsledek testu neprůkazný. Pouze u jednoho člena posádky je s určitostí prokázána nepřítomnost látky v těle.
\item[KAPITÁN] V počítači kapitána jsou jeho poslední slova, která napovídají jak najít nakaženého člena posádky. Zároveň zde popisuje, že došlo k záměně skafandrů s nakaženým členem. Ten komu byl skafandr sebrán, nemůže být tudíž nakažený.
\end{description}

\section{\texorpdfstring{Vybavení}{Vybaveni}}
\label{sec:vybaveni}

\subsection{\texorpdfstring{Vědecký modul SD15}{Vedecky modul SD15}}
\label{subsec:SD15}
Vědecký modul nese označení \uv{Science Detector, version 1.5}.

\subsection{\texorpdfstring{Taktický modul TC33}{Taktický modul TC33}}
\label{subsec:TC33}
Taktický modul nese označení \uv{Tactical Computer, version 3.3}. Jedná se o jednoduché zařízení s jedním dvoumístným segmentovým displejem a 4 diodami. Diody jsou postupně popsány:
\begin{itemize}
\item A-Z
\item 0-9
\item OK
\item X
\end{itemize}
Zařízení se připojuje pomocí vyvedeného kabelu, který slouží také k napájení. Při připojení se zobrazují následující informace:
\begin{itemize}
\item Skříň se skafandry:\\
Po stisku tlačítka na kódovém zámku se zobrazí buď 0 a rozsvítí led \uv{X} nebo se zobrazí pořadí čísla v kódu a rozsvítí led \uv{OK}.
\item Kajuta a sklad: \\
Po připojení zobrazuje index pozice čísla, které má být stisknuto. Index odpovídá matici čísel. První je číslo řádku, druhé je číslo sloupce. Svítí led pro číslice. V případě stisku správného čísla se zobrazí \uv{OK} a nový index. Při stisku špatného čísla se zobrazí index prvního čísla nebo samé nuly až do resetu zámku.
\end{itemize}

\subsection{\texorpdfstring{Technický modul FT51}{Technický modul FT51}}
\label{subsec:FM51}
Technický modul nese označení \uv{Fixing Tool, version 5.1}. Jde o elektromagnetickou pistoli s nástavci, kterou lze vytahovat klíny (nebo nýty) z definovaných montážních otvorů a otvírat tak překážky. Nástavce jsou dvojího druh:
\begin{itemize}
\item Čtvercový profil - s tím TECHNIK začíná. Lze s ním otevřít základní překážky, jako například průchozí otvor mezi skladem a chodbou.
\item Trojúhelníkový profil - ten musí TECHNIK získat. Lze se sním dostat za významné překážky, jako například při odpojování SAMa.
\end{itemize}

\subsection{\texorpdfstring{Příručka poradce}{Prirucka poradce}}
\label{subsec:prirucka_poradce}

\subsection{\texorpdfstring{Letové instrukce}{Letove instrukce}}
\label{subsec:letove_instrukce}

\begin{quote}
LETOVÉ INSTRUKCE DO SYSTÉMU DENEB
\begin{enumerate}
\item Let zahájíte na souřadnicích AD11:2F28:0070, tj. na planetě Zemi. Zde bude naložena posádka a materiál do skladu.
\item Po opuštění těchto souřadnic se sklad automaticky zamyká na základě vnitřního bezpečnostního předpisu. Posádka nebude vybavení potřebovat, jelikož v době cesty do systému DENEB bude v hibernaci. Očekávaná doba cesty je 2 týdny, 6 dnů a 15 hodin.
\item Vybavení skladu bude opět k dispozici na souřadnicích systému DENEB 3E23:089B:8AB4. Zde se očekává výsadek posádky. Doba trvání výsadku a průzkumu je předpokládána na 2 dny.
\item Po získání vzorků z tělesa planetky se tyto uloží do boxu v laboratoři. Loď opustí systém DENEB.
\item Sklad bude opět automaticky uzamknut a posádka stráví návrat mimo hibernaci. Zásoby potravin z replikátoru jsou dostatečné minimálně na 1 měsíc pro celou posádku.
\end{enumerate}
\end{quote}

\subsection{\texorpdfstring{Nouzový vysílač SOS}{Nouzovy vysilac SOS}}
\label{sec:SOS}
SOS souprava slouží pro komunikaci mezi hráči a správcem místnosti. Skrze tuto soupravu lze předávat nápovědu. V ději jde nouzový vysílač-přijímač, který má jen krátký dosah a lze se s ním spojit jen s loděmi, které jsou nablízku. V době probuzení je nedaleko (měsíc cesty) loď MERKUR. Pokud je SOS aktivován, pak lze získávat rady od správce místnosti.

POZNÁMKY:
\begin{itemize}
\item Posádka MERKURu není schopna určit polohu DIONÝSa. Zná však svojí polohu.
\end{itemize}

\section{\texorpdfstring{Příkazy}{Prikazy}}
\label{sec:prikazy}
Všechny příkazy jsou pod správou SAMa. Po odpojení SAMa se pokládají všechny za zrušené.

\subsection{\texorpdfstring{S právem taktického důstojníka}{S pravem taktickeho dustojnika}}
\label{subsec:s_pravem_TD}
\begin{itemize}
\item[D6] INVENTURA SKLADU - vyhlášení inventury ve skladu odemyká skříňky s vybavením.
\end{itemize}
\subsection{\texorpdfstring{S právem kapitána}{S pravem kapitana}}
\label{subsec:s_pravem_KP}
Tyto příkazy může rušit nebo zadávat taktický důstojník až po ověření smrti kapitána.
\begin{itemize}
\item[C1] KARANTÉNA - vyhlášení karantény pro celou loď. Nelze používat únikový modul.
\item[C2] EVAKUACE - rozsvítí únikové východy, spustí evakuační protokol.
\end{itemize}

\subsection{\texorpdfstring{Speciální}{Speciálni}}
\label{subsec:specialni}
\begin{itemize}
\item[BY] BOUNTY - tímto příkazem přebere TAKTICKÝ DŮSTOJNÍK pravomoce kapitána. 
\end{itemize}

\section{\texorpdfstring{Plakáty}{Plakaty}}
\label{sec:plakaty}

\subsection{\texorpdfstring{Mapa lodi}{Mapa lodi}}
\label{subsec:mapa_lodi}
Plakát slouží ke správnému pojmenování prostorů v lodi. Visí ve skladu, předsálí můstku a v kapitánově kajutě. Názvy jsou česky a anglicky.

\subsection{\texorpdfstring{Zákony robotiky}{Zakony robotiky}}
\label{subsec:zakony_robotiky}
Plakát je návodem pro hráče, jak jednat se SAMem. SAM totiž nesmí ublížit lidem, ani dopustit, aby jim bylo ublíženo. Zároveň bude žádat, aby jej vzala posádka do únikového modulu. Nápis na plakátě musí obsahovat následující ve obou jazycích:

\begin{quote}
\uv{Isaac Asimov} \\
\uv{1) Robot nesmí ublížit člověku nebo svou nečinností dopustit, aby bylo člověku ublíženo.} \\
\uv{2) Robot musí uposlechnout příkazů člověka, kromě případů, kdy jsou tyto příkazy v rozporu s prvním zákonem.} \\
\uv{3) Robot musí chránit sám sebe před poškozením, kromě případů, kdy je tato ochrana v rozporu s prvním nebo druhým zákonem.}  
\end{quote}

\subsection{\texorpdfstring{Požární směrnice}{Pozarni smernice}}
\label{subsec:pozarni_smernice}
Plakát visí na můstku a je návodem, jak se dostat do kapitánovi kajuty. Nápis na plakátě musí obsahovat následující:

\begin{quote}
\uv{Při zpozorování požáru použijte hasicí přístroj dle návodu na hasicím přístroji.} \\
\uv{Požár vyhlaste voláním hoří a podle možností zajistěte neprodlené informování kapitána a zbytku posádky.} \\
\uv{Místnosti s hlášením požáru budou automaticky zpřístupněny pro případný zásah bez ohledu na přístupová oprávnění.} \\
\uv{Proveďte nutná bezpečností opatření pro zamezení šíření požáru.} \\
\uv{Při úniku postupujte dle únikového značení.} 
\end{quote}

\section{\texorpdfstring{Řešení úkolů}{Reseni ukolu}}
\label{sec:reseni_ukolu}
\begin{center}
	\begin{tabular}{l | l | l | l | l}
	\hline \hline
		1. & TD & PO & TE & otevření únikové kapsle \\ \hline
		2. & TD & PO & PI & otevření skříněk \\ \hline
		3. & TD & PO & VE & identifikace infikovaného člena \\ \hline
		4. & TD & TE & PI & doplnění zásoby vzduchu do únikového modulu \\ \hline
		5. & TD & TE & VE & únik ze skladu \\ \hline
		6. & TD & PI & VE & otevření kajuty \\ \hline
		7. & PO & TE & PI & zjištění souřadnic \\ \hline
		8. & PO & TE & VE & otevření můstku \\ \hline
		9. & PO & PI & VE & zajištění potravin \\ \hline
		10. & TE & PI & VE & napájení energií únikového modulu \\ \hline \hline
	\end{tabular}
\end{center}

\subsection{\texorpdfstring{Hvězdy}{Hvezdy}}
\label{subsec:vyvazeni_hry}
Možnost (pravděpodobnost) získání hvězdy v daném úkolu:
\begin{center} 
	\begin{tabular}{l | c | c | c | c | c | c | c | c | c | c }
	\hline \hline 
		   & 1. & 2. & 3. & 4. & 5. & 6. & 7. & 8. & 9. & 10. \\ \hline
		TD & 1.0& 0.3& 1.0& 1.0& 0.3& 0.3&    &    &    &     \\ \hline
		PO & 1.0& 0.3& 1.0&    &    &    & 0.3& 0.3& 1.0&     \\ \hline
		TE & 1.0&    &    & 1.0& 0.3&    & 0.3& 0.3&    & 1.0 \\ \hline
		PI &    & 0.3&    & 1.0&    & 0.3& 0.3&    & 1.0& 1.0 \\ \hline
		VE &    &    & 1.0&    & 0.3& 0.3&    & 0.3& 1.0& 1.0 \\ \hline \hline
	\end{tabular}
\end{center}

\end{document}